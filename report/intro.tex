% File: intro.tex
% Date: Tue Mar 26 18:42:16 2013 +0800
% Author: Yuxin Wu <ppwwyyxxc@gmail.com>
\section{Introduction}

\begin{enumerate}
	\item 编译:
		程序共有四个版本,分别为串行,MPI,OpenMP,pthread.需通过改变环境变量\verb|DEFINES|分别编译.源码中提供了一个脚本\verb|make_all_version|可以一次性编译出四个可执行文件.
		编译并行版本的环境变量分别为\verb|DEFINES=-DUSE_OMP, -DUSE_MPI, -DUSE_PTH|,单独编译时,可执行文件为\verb|main|
\begin{lstlisting}
$ ./make_all_version
make seq verson ...
done
make omp version ...
done
make pthread version ...
done
make mpi version ...
done
$ DEFINES=-DUSE_OMP make
[dep] main.cc ...
[dep] Gui.cc ...
[dep] Body.cc ...
[dep] utils.cc ...
[dep] common.cc ...
[dep] NBody.cc ...
[dep] NBody_Parallel.cc ...
[cc] NBody_Parallel.cc ...
[cc] NBody.cc ...
[cc] utils.cc ...
[cc] common.cc ...
[cc] Body.cc ...
[cc] Gui.cc ...
[cc] main.cc ...
Linking ...
$
\end{lstlisting}

\item 命令行参数:
	\begin{lstlisting}[basicstyle=\tiny\ttfamily]
$ ./main -h
Usage:
    ./main -b NUM [-r <switch>] [-w <switch>] [-n <NUM_OF_PROC>] [-s <SIZE>] [-t <STEP>] [-h]
Options:
    --ball=NUM,  -b      number of balls. Default: 20.
    --disp=0/1,  -d      a switch on display mode containing a big ball. Default: 1
    --wall=0/1,  -w      a switch on whether to use window border as walls. Default: 1
    --nproc=NUM, -n      number of threads(pthread only). number of CPUs by default
    --size=SIZE, -s      size of window(by pixels). Format: [width]x[height]
                         eg. 1200x800 (default)
    --step=NUM,  -t      number of loops to operate in simulation.
                         NOTE: GUI will be off to calculate time.
    --help,              -h      print help.
\end{lstlisting}
程序支持如下的命令行参数:
\begin{description}
	\item \verb|--ball=NUM|指定小球个数.
	\item \verb|--disp=0/1,--wall=0/1|演示开关与墙壁开关.
	\item \verb|--nproc=NUM|指定pthread使用的线程数量,对其他多线程模式无效.
	\item \verb|--size=SIZE|指定窗口大小.
	\item \verb|--step=NUM|指定循环次数.
	\item \verb|--help|输出帮助信息.
\end{description}

\item 测试运行:
\begin{lstlisting}
$ ./main -b 200 -t 200
0.748785 seconds in total...
$
\end{lstlisting}

\item 界面展示:
\begin{lstlisting}
$ ./main
10.636224 seconds in total...
$ ./main -d 0
14.245977 seconds in total...
\end{lstlisting}

\begin{figure}[H]
	\centering
	\includegraphics[scale=0.4]{res/screen.png}
	\includegraphics[scale=0.4]{res/screen2.png}
\end{figure}


在窗口中,可使用空格键进行暂停,鼠标左中右键点击屏幕,可查看距点击位置最近的小球信息.按ESC键退出.

\end{enumerate}
